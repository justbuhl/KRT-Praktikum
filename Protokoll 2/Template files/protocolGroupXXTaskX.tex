\documentclass{ifacconf}

\usepackage{natbib}            % you should have natbib.sty
\usepackage[utf8]{inputenc}    % Eingabe von Umlauten im Editor, dieser sollte dann auch auf utf8 Encoding eingestellt sein
\usepackage{graphicx}          % Include this line if your 
                               % document contains figures,
%\usepackage[dvips]{epsfig}    % or this line, depending on which
                               % you prefer.
                               
\usepackage{units}

% for German
% \usepackage{ngerman}           % neue Deutsche Rechtschreibung, Silbentrennung
% \usepackage[T1]{fontenc}       % Trennung mit Umlauten

% to include tikz pictures of figure created with matlab2tikz, see also file ``plotFigureTest.m''
\usepackage{tikz}
\usepackage{pgfplots}
\pgfplotsset{compat=newest}  % use newest version of pgfplots
\usepackage{amsmath}  % useful for math

% to include the legend into the caption. The commands are
%\mlLineLegend{red}
%\mlLineLegendDashed{red}
%\mlLineLegendDotted{red}
%\mlLineLegendDashDotted{red}
%\mlPointLegend{red}
\newlength{\mlLegendThickness}
\setlength{\mlLegendThickness}{0.4mm}
\newlength{\mlLegendHeight}
\setlength{\mlLegendHeight}{0.6ex}
\newcommand{\mlLineLegend}[1]{\mbox{\color{#1}
\protect\rule[\mlLegendHeight]{3mm}{\mlLegendThickness}\hspace*{-1mm}
}}
\newcommand{\mlLineLegendDashed}[1]{\mbox{\color{#1}
\protect\rule[\mlLegendHeight]{1.5mm}{\mlLegendThickness}\hspace*{0mm}
\protect\rule[\mlLegendHeight]{1.5mm}{\mlLegendThickness}\hspace*{-1mm}
}}
\newcommand{\mlLineLegendDotted}[1]{\mbox{\color{#1}
\protect\rule[\mlLegendHeight]{0.4mm}{\mlLegendThickness}\hspace*{0mm}
\protect\rule[\mlLegendHeight]{0.4mm}{\mlLegendThickness}\hspace*{0mm}
\protect\rule[\mlLegendHeight]{0.4mm}{\mlLegendThickness}\hspace*{0mm}
\protect\rule[\mlLegendHeight]{0.4mm}{\mlLegendThickness}\hspace*{-1mm}
}}
\newcommand{\mlLineLegendDashDotted}[1]{\mbox{\color{#1}
\protect\rule[\mlLegendHeight]{1.5mm}{\mlLegendThickness}\hspace*{0mm}
\protect\rule[\mlLegendHeight]{0.4mm}{\mlLegendThickness}\hspace*{0mm}
\protect\rule[\mlLegendHeight]{1.5mm}{\mlLegendThickness}\hspace*{0mm}
\protect\rule[\mlLegendHeight]{0.4mm}{\mlLegendThickness}\hspace*{-1mm}
}}
\newcommand{\mlPointLegend}[1]{\mbox{\color{#1}
\protect\rule[\mlLegendHeight]{0.4mm}{\mlLegendThickness}\hspace*{-0.75mm}
}}

\begin{document}

\begin{frontmatter}

\title{KRT Praktikum: Protokoll 2 - L2 und H2}

\thanks[footnoteinfo]{Institute for Systems Theory and Automatic Control, University of Stuttgart, Germany. \textit{http://www.ist.uni-stuttgart.de}}

% include all authors, underline corresponding author
\author{Kimon Beyer, Yves Gaßmann, Justin Buhl} 
% \author{}

\begin{abstract}        
Im folgenden Protokoll werden die Ergebnisse des zweiten Labortages sowie der zweiten Hausaufgabe im Ramen des Praktikums Konzepte der Regelungstechnik des IST behandelt. 
Dabei erfolgte die aus dem Protokoll 1 gewonnene Identifikation der fehlenden Modellparameter zur vollständigen Beschreibung des Helikopter Modells. Des Weiteren wurde 
nun dieses Modell simulativ validiert, sowie linearisiert.                       % Abstract of not more than 250 words.
\end{abstract}

\end{frontmatter}


\section{Zweiter Labortag (L2)}
Bei den folgenden Gleichungen handelt es sich um die aus dem ersten Protokoll modellierten Systemgleichungen des Helikopters: 

\begin{equation}\label{eq:alpha}
    \ddot{\alpha} = \frac{-(F_{\mathrm{vorne}} + F_{\mathrm{hinten}})  l_{\mathrm{heli}}  \sin(\gamma)}{I_{\alpha}}
\end{equation}

\begin{equation}
    \ddot{\beta} = \frac{(F_{\mathrm{vorne}} + F_{\mathrm{hinten}})  l_{\mathrm{heli}}  \cos(\gamma) - F_{\mathrm{g}}  l_{mmp}  \cos(\beta)}{I_{\beta}}
\end{equation}

\begin{equation}\label{eq:gamma}
\ddot{\gamma} = \frac{(F_{\mathrm{vorne}} - F_{\mathrm{hinten}})  l_{rotor}}{I_{\gamma}}
\end{equation}

Mit
\begin{equation}
F_{\mathrm{g}} = m_{\mathrm{mmp}}g
\label{eq:F_g}
\end{equation}

Wie bereits in Protkoll 1 beschrieben, fehlen nun nurnoch wenige Werte zur vollständigen Systemidentifikation. Hier macht sich nun der große Vorteil der CAD Modelle zu Nutzen,
da keinerleich Trägheitstensoren oder Gesamtmassen bestimmt werden müssen. Diese können direkt aus dem CAD Modell entnommen werden. Dies ist äußerst hilfreich, da die experimentelle 
Bestimmung der Trägheitstensoren sowohl aufwändig als auch fehleranfällig ist. Daher wird davon ausgegangen, dass das Modell auch in seiner bereits Vereinfachten Form bereits
eine gute Systembeschreibung des realen Systems darstellt. 

Die einzigen zu bestimmenden Systemparameter sind somit die Spannung-Kraft Kennlinien $F_{vorne}$ und $F_{hinten}$ der Rotoren. 
Dafür wurde der reale Helikopter auf \(\beta = 0\) fixiert. Die einzelnen Rotoren wurden nun einzeln geprüft, da davon auszugehen ist, dass die Rotoren unterschiedliche Kennlinien liefern könnten. 
Die Rotor-Einheit wurde nun von unten auf eine Waage gelegt und von oben mit einem Gewicht beschwert. Nun wurden verschiedene Spannungslevel angelegt und die jeweilige Differenz der Waage abgelesen.
Mit dieser Methode konnte demnach über die Differenz der Gewichtskraft ein Rückschluss auf die durch den Rotor erzeugte Auftriebskraft gezogen werden.
\begin{equation}
F = m g
\end{equation}
Dabei ist $g = 9.81\,\mathrm{m/s^2}$ die Gewichtskraft.
Die so erhaltenen Messwerte wurden in Matlab ausgewertet und eine lineare Regression durchgeführt, um die Spannung-Kraft Kennlinie zu bestimmen.

Hier die Spannungs-Kraft kennlinie einfügen: 
TODO

Beschreiben dass wir für beide Rotoren die gleiche Kraftlinine annehmen und der Regler dies kompensiert TODO

Damit wurden sind nun alle Systemparameter bekannt und das Modell kann im folgenden auf Simulink implementiert und simulativ validiert werden.
Darauf wird in folgenden eingegangen.


\section{Zweite Hausaufgabe (H2)}

Wie bereits in Protokoll 1 diskutiert, handelt es sich bereits um ein stark vereinfachtes Modell des Helikopters. Da die Auswirkung des Winkels $\beta$ auf die Beschleunigung $\ddot{\alpha}$ vernachlässigt wird,
muss im stetig in Betracht gezogen werden, diese Modellrekuktion anzupassen, sobald sich diese Vernachlässigung als problematisch herausstellt. 

Zunäcsht wurden nun die nichtlinearen Systemgleichungen \eqref{eq:alpha}--\eqref{eq:gamma} in Simulink implementiert und simulativ validiert.

Hier nun die Simulink Diagramme einfügen: TODO

Simulink Diagramme beschreiben und bewerten TODO

Des Weiteren wurden nun die Gleichungen linearisiert, und direkt in Zustandsraum gebracht. 

Dafür wurden die Systemgleichungen \eqref{eq:alpha}--\eqref{eq:gamma} in Matlab linearisiert. 

Als Arbeitspunkt wurde .... TODO

Hier nun die linearisierten Gleichungen einfügen: TODO

Zustandsraum einfügen: TODO

Die Darstellung in Zustandsraum ist essentiell, da sie sowohl die 
Grundlage für den späteren linearen Regelerentwurf darstellt, als auch Systemtheoretische Eigenschaften des Systems aufzeigt.
So liegen beispielsweise die Systempole des linearten Systems bei 

Systempole ? Systempole bewerten TODO
Beobachtbarkeit und Steuerbarkeit (können wir ja einfach sagen dass wirs geprüft haben und die Bedingungen erfüllt sind?)TODO


\section{Kommentare und Ausblick auf L3}
Hier nun beschreiben was im nächsten labor gemacht wird TODO

\bibliography{references}


\end{document}


