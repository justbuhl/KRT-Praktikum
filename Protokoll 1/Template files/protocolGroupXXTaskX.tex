\documentclass{ifacconf}

\usepackage{natbib}            % you should have natbib.sty
\usepackage[utf8]{inputenc}    % Eingabe von Umlauten im Editor, dieser sollte dann auch auf utf8 Encoding eingestellt sein
\usepackage{graphicx}          % Include this line if your 
                               % document contains figures,
%\usepackage[dvips]{epsfig}    % or this line, depending on which
                               % you prefer.
                               
\usepackage{units}

% for German
% \usepackage{ngerman}           % neue Deutsche Rechtschreibung, Silbentrennung
% \usepackage[T1]{fontenc}       % Trennung mit Umlauten

% to include tikz pictures of figure created with matlab2tikz, see also file ``plotFigureTest.m''
\usepackage{tikz}
\usepackage{pgfplots}
\pgfplotsset{compat=newest}  % use newest version of pgfplots
\usepackage{amsmath}  % useful for math

% to include the legend into the caption. The commands are
%\mlLineLegend{red}
%\mlLineLegendDashed{red}
%\mlLineLegendDotted{red}
%\mlLineLegendDashDotted{red}
%\mlPointLegend{red}
\newlength{\mlLegendThickness}
\setlength{\mlLegendThickness}{0.4mm}
\newlength{\mlLegendHeight}
\setlength{\mlLegendHeight}{0.6ex}
\newcommand{\mlLineLegend}[1]{\mbox{\color{#1}
\protect\rule[\mlLegendHeight]{3mm}{\mlLegendThickness}\hspace*{-1mm}
}}
\newcommand{\mlLineLegendDashed}[1]{\mbox{\color{#1}
\protect\rule[\mlLegendHeight]{1.5mm}{\mlLegendThickness}\hspace*{0mm}
\protect\rule[\mlLegendHeight]{1.5mm}{\mlLegendThickness}\hspace*{-1mm}
}}
\newcommand{\mlLineLegendDotted}[1]{\mbox{\color{#1}
\protect\rule[\mlLegendHeight]{0.4mm}{\mlLegendThickness}\hspace*{0mm}
\protect\rule[\mlLegendHeight]{0.4mm}{\mlLegendThickness}\hspace*{0mm}
\protect\rule[\mlLegendHeight]{0.4mm}{\mlLegendThickness}\hspace*{0mm}
\protect\rule[\mlLegendHeight]{0.4mm}{\mlLegendThickness}\hspace*{-1mm}
}}
\newcommand{\mlLineLegendDashDotted}[1]{\mbox{\color{#1}
\protect\rule[\mlLegendHeight]{1.5mm}{\mlLegendThickness}\hspace*{0mm}
\protect\rule[\mlLegendHeight]{0.4mm}{\mlLegendThickness}\hspace*{0mm}
\protect\rule[\mlLegendHeight]{1.5mm}{\mlLegendThickness}\hspace*{0mm}
\protect\rule[\mlLegendHeight]{0.4mm}{\mlLegendThickness}\hspace*{-1mm}
}}
\newcommand{\mlPointLegend}[1]{\mbox{\color{#1}
\protect\rule[\mlLegendHeight]{0.4mm}{\mlLegendThickness}\hspace*{-0.75mm}
}}

\begin{document}

\begin{frontmatter}

\title{KRT Praktikum: Protokoll 1 - L1 und H1}

\thanks[footnoteinfo]{Institute for Systems Theory and Automatic Control, University of Stuttgart, Germany. \textit{http://www.ist.uni-stuttgart.de}}

% include all authors, underline corresponding author
\author{Kimon Beyer, Yves Gaßmann, Justin Buhl} 
% \author{}

\begin{abstract}                          % Abstract of not more than 250 words.
In diesem ersten Protokoll handelt es sich um eine Zusammenfassung des Vorgehens der Gruppe AWe1 des Praktikums "Konzepte der Regelungstechnik" am Institut für Systemtheorie und Regelungstechnik der Universität Stuttgart. In diesem ersten Protokoll werden die Ergebnisse von L1 und H1 dargestellt, wobei sich der erste Teil um L1 handelt und der zweite Teil um H1. Während sich L1 mit der Betriebnahme des Versuchsstandes beschäftigt, wird im anschließenden H1 Teil die Modellierung des 3DOF Helikopters durchgeführt. Hierbei wurde auf den Newton Euler Ansatz, mithilfe des Drallsatzes zurückgegriffen.Abschließend wird ein Ausblick auf die in L2 geplante Modellidentifizierung gegeben.
\end{abstract}

\end{frontmatter}

\section{Einleitung}
In diesem Protokoll werden die Ergebnisse der ersten Laboraufgabe (L1) und der ersten Hausaufgabe (H1) des 
Praktikums "Konzepte der Regelungstechnik" am Institut für Systemtheorie und Regelungstechnik der Universität Stuttgart zusammengefasst. Ziel von L1 ist es, den Versuchsstand in Betrieb zu nehmen und die grundlegenden Funktionen zu verstehen. H1 konzentriert sich auf die Modellierung des 3DOF Helikopters (Quanser 3 DOF Helicopter \cite{Quanser2012}), um eine Grundlage für die anschließende Modellidentifizierung in L2 zu schaffen.
Das Praktikum "Konzepte der Regelungstechnik" am Institut für Systemtheorie und Regelungstechnik der Universität Stuttgart \citeauthor{IST2025} beschäftigt sich mit der Regelungstechnik anhand eines 3DOF Helikopters, welcher in den Laboraufgaben und Hausaufgaben modelliert und dessen Modell anschließend zur Regelung verwendet wird.
Beim zu regelnden Versuchsstand handelt es sich um einen 3DOF Helikopter, der entlang seiner Hauptachsen/Gelenkachsen (Schwenkwinkel, Steigwinkel, Nickwinkel) bewegt werden kann.
Die Bewegung um die Achsen wird dabei durch zwei Elektromotoren ermöglicht.
Ziel des Praktikums ist es eine vorgegebene Trajektorie innerhalb einer gewissen Zeit abzufliegen und dabei die jeweiligen 
Beschränkungen einzuhalten. Dabei wird im folgenden die Inbetriebnahme des Versuchsstandes 
beschrieben und die Modellierung des Helikopters durchgeführt. Dieses Modell wird im laufenden 
Praktikum genutzt um ein Regelerentwurf durchzuführen und am Teststand zu implementieren. 

\begin{figure}[htbp]
    \centering
    \includegraphics[width=0.8\columnwidth]{Helikopter.pdf} % Pfad/Dateiname anpassen
    \caption{Versuchsstand: 3DOF-Helikopter. Besteht aus Haupt- und Nebenarm, sowie Gegengewicht 
    und am Ende des Arms befindlichen Helikopter.}
    \label{fig:helikopter}
\end{figure}

\section{Erster Labortag (L1)}

Um den Versuchsstand in Betrieb zu nehmen wird die "Quanser Quarc" Schnittstelle, eine Erweiterung für Matlab/Simulink, verwendet.
Diese Schnittstelle ermöglicht es, die Hardware mit Matlab Simulink zu verbinden und die Motoren/Signale über die
jeweilige Blöcke anzusteuern und auszulesen. Über einen "Write Analog" Block werden die zwei 12V Motoren, 
die zur verfügung stehen, sowie der Magnet angesteuert. Der Magnet wird benötigt um das Verfrachten einer Metallkugel
als Teil der Praktikumsendaufgabe zu ermöglichen. Eine Spannungsdifferenz an den Motoren ermöglicht nun die Steuerung des
Steigwinkel und somit auch des Schwenkwinkels. Es ist zu beachte, dass diese Motoren nicht direkt angesteuert werden, sondern ein Verstärker das Signal nochmals um Faktor 4 versträkt. Somit sind Werte im Bereich von -4 bis 4 zulässig. An diesem Block kann auch der "Game Controller" angeschlossen werden, mittels dem 
der Versuchsstand manuell angesteuert werden kann. 
Zur manuellen Steuerung des Versuchsstandes wurde ein Flugjoystick (Logitech Attack-3) verwendet, der über den "Game Controller" Block in Simulink eingebunden wird.
Über die Joystick Achsen können die Spannungen für die beiden Motoren eingestellt werden, um den Helikopter zu bewegen.
Das Bewegen des Joysticks nach vorne/hinten verändert die Spannung beider Motoren gleichmäßig, und sorgt um eine Änderung des Steigwinkels in positiver/negativer Richtung.
Das Bewegen des Joysticks nach links/rechts verändert die Spannungen der beiden Motoren unterschiedlich, wodurch eine Änderung des Nickwinkels und im Zuge dessen auch eine Änderung des Schwenkwinkels in positiver/negativer Richtung ermöglicht wird. Hierzu wird die Spannung des einen Motors um Faktor 0.75 bis 1 (je nach Auslenken des Joysticks) verändert. Eine Bewegung des Joysticks nach links führt dabei zu einer positiven Änderung sowohl des Nick- als auch des Schwenkwinkels. In umgekehrter Richtung verhält es sich entsprechend.
Der Magnet wird über einen separaten Knopf am Joystick angesteuert, der eine Spannung von 12V an den Magneten legt, um die Metallkugel anzuziehen. Diese Spannung liegt an, solange der Knopf gedrückt wird.
\begin{figure}
    \centering
    \includegraphics[width=0.9\columnwidth]{Helikopter_winkel.png}
    \caption{Winkeldefinitionen des 3DOF Helikopters \cite{IST2025}}
    \label{fig:helikopter_winkel}
\end{figure}
Zum Auslesen der aktuellen Position des Helikopters werden die verbauten Encoder genutzt.
Der "Read Encoder" Block ermöglicht das Auslesen der Winkelinkremente der drei Achsen der verbauten Encoder, die relativ zur Ausgangsposition gemessen werden.
Diese Winkelinkremente müssen noch umgerechnet und mit richtigem Vorzeichen versehen werden, um die 
nötigen Konventionen von Rechtshandsystemen und der Modellbildung zu gewährleisten (vgl. Fig. \ref{fig:helikopter_winkel}). Hierzu werden die in Tabelle \ref{tab:umrechnung_encoder} dargestellten Spezifikationen der Encoder genutzt. Der Steigwinkel $\beta$ musste zusätzlich um 27° korrigiert werden, da der Nullpunkt des Encoders (Startpunkt des Helikopters)nicht mit der waagerechten Position des Helikopters übereinstimmt.

\begin{table}[htbp]
\centering
\caption{Spezifikationen der Encoder}
\label{tab:umrechnung_encoder}
\begin{tabular}{|c|c|c|}
\hline
Winkel & Encoderauflösung & Korrektur der Richtung \\ \hline
$\alpha$ & 8192 Impulse/Umdrehung & Vorzeichenwechsel \\ \hline
$\beta$  & 4096 Impulse/Umdrehung & Vorzeichenwechsel \\ \hline
$\gamma$ & 4096 Impulse/Umdrehung & Vorzeichenwechsel \\ \hline
\end{tabular}
\end{table}

Diese Blöcke werden letztendlich in 3 Simulink Subsystems unterteilt, "Input Generation", "Data acquisition" und "Plant". 
Die Plant soll dabei durch ein Modell des Helikopters ersetzen werden können. Im Subsystem "Input generation" 
der Write Digital/Analog Block sowie der Game Controller zusammengefasst. Im Subsystem "Data acquisition" werden die Encoderwerte ausgelesen und aufbereitet.

\section{Erste Hausaufgabe (H1)}
Für die Modellierung des 3DOF Helikopters wurde versucht eine möglichst einfache Modellannahme zu treffen, 
um die Komplexität des Modells gering zu halten und mit einem möglichst überschaubaren Zeitaufwand 
eine möglichst gute Abbildung des Systemverhaltens zu extrahieren. Zunächst wurde versucht,
den Versuchsstand in SolidWorks zu modellieren und mit anschließenden Matlab tools die Systemgleichungen 
zu extrahieren. Das Auslesen der Bewegungsgleichungen eine nichtlinearen Modells ist jedoch nicht möglich, wesshalb diese Idee verworfen wurde. Dennoch wurde das Modell genutzt, um Modellparameter zu extrahieren. 
Darauf wird noch im Weiteren eingegangen. Für die physikalische Modellbildung wurde nun zunächst die Grundannahme getroffen,
die einzelnen Achsen des Helikopters unabhängig voneinander zu betrachten. Die 
Dynamik der einzelnen Achsen ergibt sich dann durch das Momentengleichgewicht und dem Drallsatz.
\begin{equation}
M_{\mathrm{ges}} = M_{\mathrm{grav}} + M_{\mathrm{motor}}
\label{eq:momentengleichgewicht}
\end{equation}
\begin{equation}
M = I \dot{\omega}
\label{eq:rotational_dynamics}
\end{equation}
Dabei bezeichnet $M_{\mathrm{ges}}$ das gesamte Moment, $M_{\mathrm{grav}}$ das durch die Gravitationkraft verursachte Moment
und $M_{\mathrm{motor}}$ das durch die Motoren erzeugte Moment. $I$ ist das Trägheitsmoment der jeweiligen Achse und $\dot{\omega}$ die jeweilige
Winkelbeschleunigung um die betrachtete Achse. Die Dynamik der drei Rotationen um die jeweilige Achse ergibt sich
letztendlich durch zweifache Integration der Winkelbeschleunigung. 

Für jede Rotationsachse wird angenommen, dass sich ein Starrkörper um diese bewegt. Dies vereinfacht die Modellbildung, sorgt jedoch in diesem Fall für nur kleine Änderungen der Trägheitstensoren. Das heist für den Nickwinkel um die Nickachse, bewegt sich lediglich
der Helikopter als Starrkörper (vgl. Fig. \ref{fig:bild1}). Für die Bewegung um die Steigachse, wird der Helikopter sowie der gesamte Arm (d.h Hauptarm, Magnethalterung, Nebenarm und Gegengewicht) als Starrkörper mit lediglich einem Freiheitsgrad um die Steigachse (vgl. Fig. \ref{fig:bild2}). Für die Berechnung der Trägheitsmomente wurde ein Nickwinkel von 0° angenommen.  
Für die Bewegung um die Schwenkachse, betrachten wir nun den Helikopter, Arm und einen zusätzlichen vertikalen Arm, 
alle drei Teile als ein starrer Starrkörper mit einem Freiheitsgrad um die Schwenkachse (vgl. Fig. \ref{fig:bild3}). 

\begin{figure}[htbp]
    \centering
    \includegraphics[width=0.9\columnwidth]{heli.png}
    \caption{CAD Modell des Helikopters}
    \label{fig:bild1}
\end{figure}

\begin{figure}[htbp]
    \centering
    \includegraphics[width=0.9\columnwidth]{arm_heli.png}
    \caption{CAD Modell des Helikopters mit Arm}
    \label{fig:bild2}
\end{figure}

\begin{figure}[htbp]
    \centering
    \includegraphics[width=0.9\columnwidth]{arm_heli_staender.png}
    \caption{CAD Modell des Helikopters mit Arm und Ständer.}
    \label{fig:bild3}
\end{figure}


Des Weiteren wurden folgende Annahmen getroffen:
\begin{itemize}
    \item Reibungen sowie Luftwiderstände werden vernachlässigt.
    \item Der Schwerpunkt des Helikopters selbst liegt im Rotationspunkt der Drehachse. Diese Symmetrie
    hat zur Folge, dass für die Pitch Achse kein Gravitationsmoment berücksichtigt werden müssen, da sich 
    die Gewichtskräfte gegenseitig aufheben.

\end{itemize}
Zur Extrahierung der Massenträgheitsmomente wurden die in Abbildungen~\ref{fig:bild1},~\ref{fig:bild2} und~\ref{fig:bild3} gezeigten CAD-Modelle aus SolidWorks genutzt.
Es ist wichtig zu erwähnen, dass die jeweiligen Massenträgheiten in den jeweiligen Koordinatensystemen der Rotationsachsen extrahiert wurden sind.
Dabei wurden folgende Massenträgheiten extrahiert sowie mit folgenden Längen als
Hebelarme des Starrkörpers festgelegt:

\begin{table}[htbp]
\centering
\caption{Modelparameter}
\label{tab:modelparams}
\begin{tabular}{|c|c|p{4.5cm}|}
\hline
Variable & Wert & Beschreibung \\ \hline
$I_{\alpha}$ & 1.130985 [kg/m$^2$] & Trägheitsmoment um die Travel-Achse \\ \hline
$I_{\beta}$  & 1.125115 [kg/m$^2$] & Trägheitsmoment um die Elevation-Achse \\ \hline
$I_{\gamma}$ & 0.040229 [kg/m$^2$] & Trägheitsmoment um die Pitch-Achse. \\ \hline
$m_{\mathrm{mmp}}$ & 3.960 [kg] & Gesamte Masse des 3DOF Helikopters mit Hauptarm \\ \hline
\end{tabular}
\end{table}

\begin{table}[htbp]
\centering
\caption{Gegebene Längen}
\label{tab:modellengths}
\begin{tabular}{|c|c|p{5cm}|}
\hline
Variable & Wert & Beschreibung \\ \hline
$l_{heli}$ & 0.655 [m] & Länge vom Aufhängepunkt zur Rotorbaugruppe \\ \hline
$l_{rotor}$  & 0.1775 [m] & Länge vom Drehpunkt der Rotorbaugruppe zum Rotor \\ \hline
$l_{mmp}$ & 0.010029 [m] & Länge von Aufhängepunkt zum Massenmittelpunkt \\ \hline
\end{tabular}
\end{table}

Bei den jeweiligen Trägheiten $$I_{\alpha}, I_{\beta}, I_{\gamma}$$ die aus dem SolidWorks Modell extrahiert 
worden sind, handelt es sich um die addierten Trägheitsmomente der, für die jeweilige Achsrotation relevante, angenommenen Starrkörper. 

Letztendlich ergeben sich folgende Differentialgleichungen für die drei Achsen:

\begin{equation}\label{eq:alpha}
    \ddot{\alpha} = \frac{-(F_{\mathrm{vorne}} + F_{\mathrm{hinten}})  l_{\mathrm{heli}}  \sin(\gamma)}{I_{\alpha}}
\end{equation}

\begin{equation}
    \ddot{\beta} = \frac{(F_{\mathrm{vorne}} + F_{\mathrm{hinten}})  l_{\mathrm{heli}}  \cos(\gamma) - F_{\mathrm{g}}  l_{mmp}  \cos(\beta)}{I_{\beta}}
\end{equation}

\begin{equation}
\ddot{\gamma} = \frac{(F_{\mathrm{vorne}} - F_{\mathrm{hinten}})  l_{rotor}}{I_{\gamma}}
\end{equation}

Mit
\begin{equation}
F_{\mathrm{g}} = m_{\mathrm{mmp}}g
\label{eq:F_g}
\end{equation}
wobei $m_{\mathrm{ges}}$ die Gesamtmasse des betrachteten Starrkörpers und $g\approx 9.81\ \mathrm{m/s^2}$ die Erdbeschleunigung ist.

In Gleichung \ref{eq:alpha} wird auf die Berücksichtigung des Steigwinkels $\beta$ verzichtet, da angenommen wird, dass dieser Winkel nur geringe Auslenkungen erfährt und somit der Einfluss auf die Travel Achse vernachlässigbar ist ($\beta$ geht im Cosinus in die Gleichung ein). Falls sich herrausstellen sollte, dass diese Annahme nicht haltbar ist, kann dies jdeoch noch angepasst werden.

\section{Kommentare und Ausblick auf L2}
Obwohl die Modellierung des Helikopters viele Vereinfachungen und Annahmen beinhaltet, erscheinen die extrahierten Modellgleichungen 
plausibel. Als besonders nützlich erwies sich die Extrahierung der Massenträgheiten aus dem CAD Modell. Hierdurch konnte eine aufwendige Handberechnung der Trägheitsmomente vermieden werden und ungenauigkeiten durch vereinfachte Annahmen über die Geometrie der einzelnen Komponenten reduziert werden.
In Ausblick auf L2 fehlt nun lediglich die Bestimmung der Motorkennlinien der beiden Motoren, um die Modellierung des Helikopters abzuschließen. Hierzu sind in L2 eine Reihe von Versuchen geplant, wobei mithilfe einer Waage und einer großen Masse die Auftriebskraft der Motoren in Abhängigkeit der angelegten Spannung gemessen werden soll. Diese Daten werden dann genutzt, um die Motorkennlinien zu bestimmen und in das Modell zu integrieren.

% Literaturverzeichnis: entries are stored in the external BibTeX file `references.bib`
% \section*{Literaturverzeichnis}

\bibliography{references}


\end{document}


